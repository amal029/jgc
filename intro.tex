\section{Introduction}
\label{sec:introduction}

Synchronous programming languages~\cite{berry92} allow building correct
by construction safety critical hard real-time applications, because
they are based on formal mathematical semantics, which allows the
compiler itself to act as a
model-checker~\cite{jagadeesan1995safety}. The major benefit of the
synchronous \textit{Model of Computation} (MoC) is that it is based on
the concept of a \textit{Finite State Machine} (FSM). Languages such as
Esterel~\cite{berry92} and its derivation SystemJ~\cite{amal10} have
been designed to reduce the state space explosion problem exhibited
during formal verification, by adhering to a very strict programmer
specified state demarcation approach. The programming model for these
languages can be described as follows: a synchronous program is
quiescent until one or more events occur from the environment. Upon
detecting the event, the synchronous program \textit{reacts}
\textit{instantaneously} in zero time to respond to these inputs and
becomes quiescent until the next input events arrive. The start and the
end of this reaction transition demarcate states of the program -- note
that internal data updates do not lead to change in the program state
thereby reducing the state space explosion problem. Furthermore, the
reaction being logically in zero-time is by definition \textit{atomic}
and always faster than the incoming input events, thereby guaranteeing
that none of the incoming events are missed. In reality, the reaction
does take time, one needs to find out the \textit{Worst Case Reaction
  Time} (WCRT) of any given synchronous program. This WCRT value
determines the shortest inter-arrival time for input events from the
environment.

Many techniques exist for computing WCRT values for synchronous programs
providing support for data-computation via a non-managed language
(primarily `C')~\cite{boldt07,proop10}. These WCRT estimation approaches
suffer from a fundamental problem; they cannot incorporate complex
data-computations within the WCRT analysis framework, because of the
many undefined behaviors, type unsafety and the general lack of formal
operational semantics inherent to low-level non-managed
languages. Hence, in this paper we are interested in the WCRT analysis
of synchronous programs that support data computation via
managed-languages like Java, which provides a well defined operational
semantics. The key hurdle to the adoption of managed languages in the
safety critical hard real-time application domain is \textit{Garbage
  Collection} (GC) for automatic memory management, since worst case
execution time estimates for real-time tasks with GC are very
pessimistic~\cite{puffitsch2013design}.

% On the one hand GC guarantees no memory leaks, thereby providing
% functional safety, but on the other hand, it has been
% shown~\cite{puffitsch2013design} that using even a state of the art
% real-time GC results in very pessimistic worst case execution time
% estimates for real-time tasks.,~\footnote{The worst case execution time
%   was estimated to be in seconds due to the GC, but the observed time
%   was in milli-seconds.} compared to the observed task execution times,
% thereby rendering worst case execution/reaction time analysis unfeasible
% in the general case with a GC.

The main \textbf{contribution} presented in this paper is a new memory
management approach that is amicable to static WCRT analysis of
synchronous programs intertwined with managed runtime environments for
data-computation. More concretely our contributions can be refined as
follows:

\begin{itemize}
\item \textit{Programming model inspired memory organization}: In this
  paper we present a new memory organization for Java and an
  accompanying static WCRT analysis based on the synchronous MoC.
\item \textit{Compiler supported memory allocation}: We present the
  compiler transformations that are needed to statically guarantee the
  \textit{Worst Case Memory Consumption} (WCMC) and \textit{allocation}.
\item \textit{Real-time analyzable back-end code generation and garbage
    collection}: We present a strategy to replace the object allocation
  bytecodes with real-time analyzable alternatives. Furthermore, we also
  present an object placement strategy, which allows for O(1) heap
  accesses in all cases.
\end{itemize}

We use the SystemJ~\cite{amal10} programming language, because it
provides the synchronous MoC as a subset along with Java data-driven
computation support. Furthermore, there exist efficient and real-time
analyzable execution architectures~\cite{nadeem2011rjop} and WCRT
analysis tools~\cite{LiMS14} that support subset of SystemJ programs;
those without GC invocation.


% Local Variables:
% mode: latex
% TeX-master: "jgc"
% End:
