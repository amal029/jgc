
\subsection{AGRC analysis}
\label{sec:agrc-analysis}

In order to statically determine the worst case memory consumption of a
program targeting TP-JOP, one needs to analyze codes generated from both
CNs and JDNs. The compiler is able to trivially determine the maximum
memory footprint required for executing CN codes from any given SystemJ
programs [] since there is no dynamic memory allocation required for
executing this code. On the other hand, the data-flow analysis for JDNs
is more intricate since it needs to analyze the original AGRC in order
to retrieve the dependency edges, which are lost after splitting CNs and
JDNs from the AGRC. We devote this section to illustrate this analysis
technique.


\begin{algorithm}[t!]
	\DontPrintSemicolon
	\SetKw{Let}{let}
	\SetKw{Fn}{Function}
	\SetAlgoNoEnd
	\SetKwFunction{KwCol}{collect\_cd\_nodes}
	\SetKwFunction{KwCrNode}{create\_node}
	\SetKwFunction{KwFindFork}{find\_matching\_fork}
	\KwIn{$AGRC$ : $(V,E)$}
	\KwOut{$mList$: a set of set of $mNode$}
	\KwData{$V_{cd}$: a set of set of clock-domain nodes}
	\KwData{$mNode$ = [$Name$:string,$Must$:$mNode$,$May$:set of $mNode$] }
	\tcc{Grouping nodes for each clock-domain}
	\Let{$V_r = \emptyset \cup V$}\;
	\Let{$V_{cd} = \emptyset$}\;
	\ForEach{$v \in V$}{
		\lForEach{$(v_i,v_j) \in E$}{\lIf{$v_j = v$}{$V_r = V_r \backslash \{v\}$}}
	}
	\lForEach{$v \in V_r$}{$V_{cd} = V_{cd} \cup$ \{\KwCol{$v,E$}\}}\;
	\tcc{Perform May and Must analysis}
	\ForAll{$V \in V_{cd}$}{
		\Let $V_{temp} = \emptyset$\;
		\ForEach{$v \in V$}{
			$V_{temp} = V_{temp} \cup$ \KwCrNode{$v,E$}
		}
		$mList = mList \cup \{V_{temp}\}$
	}
	\;
	\Fn \KwCrNode{$v,E$}:\;
	\Begin{
		\Let{$E_i = \lambda(v)$}\;
		\If{$v = ActionNode$}{
			\If{$|E_i| = 1$}{
				\Let{$\{(v_i,v_j)\} = E_i$}\;
				\Let $newNode = [Name \leftarrow get\_name(v_j); Must
				\leftarrow \KwCrNode{$v_i,E$}; May \leftarrow Null]$\;
				\Return{newNode}
			}
			\ElseIf{$|E_i| > 1$}{
				\Let $V_{temp} = \emptyset$\;
				\ForEach{$(v_i,v_j) \in E_i$}{
					$V_{temp} = V_{temp} \cup \{\KwCrNode{$v_i,E$}\}$
				}
				\Return{$[Name \leftarrow get\_name(v_j); Must
				\leftarrow Null ; May \leftarrow V_{temp}]$}
			}
			\ElseIf{$|E_i| = 0 $}{
				\Return{$[Name \leftarrow get\_name(v_j); Must
				\leftarrow Null; May \leftarrow Null]$}
			}
		}
		\ElseIf{$v =JoinNode$}{
			\Let{$v = $ \KwFindFork{$v$}}\;
			\Let{$(v_i,v_j) = \lambda(v)$}\;
			\Return{\KwCrNode{$v_i,E$}}
		}
		\Else{
			\If{$|E_i| = 0$}{\Return{Null}}
			\ElseIf{$|E_i| = 1$}{\Return{\KwCrNode{$v_i,E$}}}
		}
	}
	\;
	\Fn \KwCol{$v,E$}:\;
	\Begin{
		\Let $V_{temp} = \emptyset$\;
		\ForEach{$(v_i,v_j) \in E$}{\If{$v = v_i$}{
			$V_{temp} = V_{temp} \cup \{v_j\}$\;
			$V_{temp} = V_{temp} \cup$ \KwCol{$v_j,E$}
		}}
		\Return{$V_{temp}$}
	}
	\caption{May and Must analysis}
	\label{alg:agrc}
	
\end{algorithm}

AGRC is a \textit{directed graph} of tuple $G =(V,E)$, where $V$ is the
set of vertices (nodes of AGRC) and $E$ is the set of ordered pairs of
vertices. Each edge $e = (v_i,v_j), \forall e \in E, v_i \in V, v_j \in
V$ is a tuple denoting that the edge is directed from $v_i$ to $v_j$.
Then a lambda  function $\lambda:v \rightarrow E_i$, $E_i \subseteq E$,
maps \textit{each} vertex to a set of edge(s) where $\forall (v_i,v_j)
\in E_i, v_j = v$ and $E_i$ can be $\emptyset$. When $|E_i| = 1$ we call
$v_i$ \textit{must} be called before $v$ whereas when $|E_i| > 1$ we
call any $v_i \in E_i$ \textit{may} be called before $v$.

The algorithm for obtaining \textit{may} and \textit{must} relationships
between the AGRC nodes is shown in Algorithm~\ref{alg:agrc}. The input
to this algorithm is an AGRC, which is a tuple $(V,E)$ as explained
previously. The first part of the algorithm, lines 1-5, groups all the
nodes, which are belonging to the same clock-domain. For instance, lines
3-4 initializes a set $V_r$, which consists of a root node of all
clock-domains in AGRC. The algorithm then traverses every clock-domain
graph, starting from these root nodes via a recursive function
\texttt{collect\_cd\_nodes($v,E$)} (line 5). A result is a set of set of
AGRC nodes $V_{cd}$. Next \textit{May} and \textit{Must} analysis is
performed (lines 6-10). Every node in $V_{cd}$ is given as an input to
the function \texttt{create\_node($v,E$)} (line 9), which traverses the
graph backward starting from the input until it reaches the root node.
Function \texttt{create\_node} first maps the input node to a set of
edges $E_i$ using the lambda function (line 14). It is then divided into
three parts:

\begin{enumerate}

	\item If the current node type is an \texttt{Action} node (line 15)
		and

		\begin{enumerate}

			\item a number of element in $E_i$ is 1, then the only parent node
				$v_i$ \textit{Must} be called before $v_j$. Then a new node type
				of $mNode$ is created with its field $Must$ initialized to the
				result of recursive call of \texttt{create\_note}.

			\item a number of element in $E_i$ is greater than 1, then any
				parent nodes $v_i \in E_i$ \textit{May} be called before $v_j$.
				A set of $mList$, consists of return results of
				\texttt{create\_node} from each parent node $v_i \in E_i$, is
				assigned to the field $May$ of a newly created $mNode$. 

			\item a number of element in $E_i$ is 0, then a new $mNode$ is
				created with both its $Must$ and $May$ fields initialized to
				$Null$.

		\end{enumerate}

	\item If the current node type is \texttt{Join} node, this means that
		there are two or more parallel reactions forked at the \texttt{Fork}
		node, which \textit{Must} be found in one of the recursive parents
		of the current node. Hence, the algorithm must continue the backward
		traversal from this \texttt{Fork} node. During compilation, the
		compiler has already constructed a Hashtable which maps each
		\texttt{Join} node to its corresponding \texttt{Fork} node as shown
		in line 28. 

	\item If the current node type is other than \texttt{Action} node (line
		31) and

		\begin{enumerate}

			\item a number of parent node ($|E_i|$) is 0 , then \textit{Null}
				is returned

			\item a number of parent node is 1, then a result of
				\texttt{create\_node} is returned.

		\end{enumerate}

\end{enumerate}

It should be noted that if the current node is other than
\texttt{Action} node, only \texttt{Join} node can have more than one
parent. Therefore, there is no need for the algorithm to consider the
case $|E_i| > 1$ (lines 32-25) when the node type is neither
\texttt{Action} nor \texttt{Join} node.


%Give formal definition of may and must. Then describe the algorithm that
%given a method, the callee method, gives the may and must caller
%methods.
































